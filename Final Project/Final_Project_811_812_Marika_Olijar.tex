% Options for packages loaded elsewhere
\PassOptionsToPackage{unicode}{hyperref}
\PassOptionsToPackage{hyphens}{url}
%
\documentclass[
  english,
  man]{apa6}
\usepackage{amsmath,amssymb}
\usepackage{lmodern}
\usepackage{ifxetex,ifluatex}
\ifnum 0\ifxetex 1\fi\ifluatex 1\fi=0 % if pdftex
  \usepackage[T1]{fontenc}
  \usepackage[utf8]{inputenc}
  \usepackage{textcomp} % provide euro and other symbols
\else % if luatex or xetex
  \usepackage{unicode-math}
  \defaultfontfeatures{Scale=MatchLowercase}
  \defaultfontfeatures[\rmfamily]{Ligatures=TeX,Scale=1}
\fi
% Use upquote if available, for straight quotes in verbatim environments
\IfFileExists{upquote.sty}{\usepackage{upquote}}{}
\IfFileExists{microtype.sty}{% use microtype if available
  \usepackage[]{microtype}
  \UseMicrotypeSet[protrusion]{basicmath} % disable protrusion for tt fonts
}{}
\makeatletter
\@ifundefined{KOMAClassName}{% if non-KOMA class
  \IfFileExists{parskip.sty}{%
    \usepackage{parskip}
  }{% else
    \setlength{\parindent}{0pt}
    \setlength{\parskip}{6pt plus 2pt minus 1pt}}
}{% if KOMA class
  \KOMAoptions{parskip=half}}
\makeatother
\usepackage{xcolor}
\IfFileExists{xurl.sty}{\usepackage{xurl}}{} % add URL line breaks if available
\IfFileExists{bookmark.sty}{\usepackage{bookmark}}{\usepackage{hyperref}}
\hypersetup{
  pdftitle={Clan, Class or State (Media)?: Evaluating the Roots of Kazakh Bride Abduction},
  pdfauthor={Marika Olijar1},
  pdflang={en-EN},
  pdfkeywords={clan, media, rhetoric, consent, post-Soviet, nationalism, re-traditionalization, class},
  hidelinks,
  pdfcreator={LaTeX via pandoc}}
\urlstyle{same} % disable monospaced font for URLs
\usepackage{graphicx}
\makeatletter
\def\maxwidth{\ifdim\Gin@nat@width>\linewidth\linewidth\else\Gin@nat@width\fi}
\def\maxheight{\ifdim\Gin@nat@height>\textheight\textheight\else\Gin@nat@height\fi}
\makeatother
% Scale images if necessary, so that they will not overflow the page
% margins by default, and it is still possible to overwrite the defaults
% using explicit options in \includegraphics[width, height, ...]{}
\setkeys{Gin}{width=\maxwidth,height=\maxheight,keepaspectratio}
% Set default figure placement to htbp
\makeatletter
\def\fps@figure{htbp}
\makeatother
\setlength{\emergencystretch}{3em} % prevent overfull lines
\providecommand{\tightlist}{%
  \setlength{\itemsep}{0pt}\setlength{\parskip}{0pt}}
\setcounter{secnumdepth}{-\maxdimen} % remove section numbering
% Make \paragraph and \subparagraph free-standing
\ifx\paragraph\undefined\else
  \let\oldparagraph\paragraph
  \renewcommand{\paragraph}[1]{\oldparagraph{#1}\mbox{}}
\fi
\ifx\subparagraph\undefined\else
  \let\oldsubparagraph\subparagraph
  \renewcommand{\subparagraph}[1]{\oldsubparagraph{#1}\mbox{}}
\fi
% Manuscript styling
\usepackage{upgreek}
\captionsetup{font=singlespacing,justification=justified}

% Table formatting
\usepackage{longtable}
\usepackage{lscape}
% \usepackage[counterclockwise]{rotating}   % Landscape page setup for large tables
\usepackage{multirow}		% Table styling
\usepackage{tabularx}		% Control Column width
\usepackage[flushleft]{threeparttable}	% Allows for three part tables with a specified notes section
\usepackage{threeparttablex}            % Lets threeparttable work with longtable

% Create new environments so endfloat can handle them
% \newenvironment{ltable}
%   {\begin{landscape}\begin{center}\begin{threeparttable}}
%   {\end{threeparttable}\end{center}\end{landscape}}
\newenvironment{lltable}{\begin{landscape}\begin{center}\begin{ThreePartTable}}{\end{ThreePartTable}\end{center}\end{landscape}}

% Enables adjusting longtable caption width to table width
% Solution found at http://golatex.de/longtable-mit-caption-so-breit-wie-die-tabelle-t15767.html
\makeatletter
\newcommand\LastLTentrywidth{1em}
\newlength\longtablewidth
\setlength{\longtablewidth}{1in}
\newcommand{\getlongtablewidth}{\begingroup \ifcsname LT@\roman{LT@tables}\endcsname \global\longtablewidth=0pt \renewcommand{\LT@entry}[2]{\global\advance\longtablewidth by ##2\relax\gdef\LastLTentrywidth{##2}}\@nameuse{LT@\roman{LT@tables}} \fi \endgroup}

% \setlength{\parindent}{0.5in}
% \setlength{\parskip}{0pt plus 0pt minus 0pt}

% Overwrite redefinition of paragraph and subparagraph by the default LaTeX template
% See https://github.com/crsh/papaja/issues/292
\makeatletter
\renewcommand{\paragraph}{\@startsection{paragraph}{4}{\parindent}%
  {0\baselineskip \@plus 0.2ex \@minus 0.2ex}%
  {-1em}%
  {\normalfont\normalsize\bfseries\itshape\typesectitle}}

\renewcommand{\subparagraph}[1]{\@startsection{subparagraph}{5}{1em}%
  {0\baselineskip \@plus 0.2ex \@minus 0.2ex}%
  {-\z@\relax}%
  {\normalfont\normalsize\itshape\hspace{\parindent}{#1}\textit{\addperi}}{\relax}}
\makeatother

% \usepackage{etoolbox}
\makeatletter
\patchcmd{\HyOrg@maketitle}
  {\section{\normalfont\normalsize\abstractname}}
  {\section*{\normalfont\normalsize\abstractname}}
  {}{\typeout{Failed to patch abstract.}}
\patchcmd{\HyOrg@maketitle}
  {\section{\protect\normalfont{\@title}}}
  {\section*{\protect\normalfont{\@title}}}
  {}{\typeout{Failed to patch title.}}
\makeatother
\keywords{clan, media, rhetoric, consent, post-Soviet, nationalism, re-traditionalization, class\newline\indent Word count: 1241}
\DeclareDelayedFloatFlavor{ThreePartTable}{table}
\DeclareDelayedFloatFlavor{lltable}{table}
\DeclareDelayedFloatFlavor*{longtable}{table}
\makeatletter
\renewcommand{\efloat@iwrite}[1]{\immediate\expandafter\protected@write\csname efloat@post#1\endcsname{}}
\makeatother
\usepackage{lineno}

\linenumbers
\usepackage{csquotes}
\ifxetex
  % Load polyglossia as late as possible: uses bidi with RTL langages (e.g. Hebrew, Arabic)
  \usepackage{polyglossia}
  \setmainlanguage[]{english}
\else
  \usepackage[main=english]{babel}
% get rid of language-specific shorthands (see #6817):
\let\LanguageShortHands\languageshorthands
\def\languageshorthands#1{}
\fi
\ifluatex
  \usepackage{selnolig}  % disable illegal ligatures
\fi
\newlength{\cslhangindent}
\setlength{\cslhangindent}{1.5em}
\newlength{\csllabelwidth}
\setlength{\csllabelwidth}{3em}
\newenvironment{CSLReferences}[2] % #1 hanging-ident, #2 entry spacing
 {% don't indent paragraphs
  \setlength{\parindent}{0pt}
  % turn on hanging indent if param 1 is 1
  \ifodd #1 \everypar{\setlength{\hangindent}{\cslhangindent}}\ignorespaces\fi
  % set entry spacing
  \ifnum #2 > 0
  \setlength{\parskip}{#2\baselineskip}
  \fi
 }%
 {}
\usepackage{calc}
\newcommand{\CSLBlock}[1]{#1\hfill\break}
\newcommand{\CSLLeftMargin}[1]{\parbox[t]{\csllabelwidth}{#1}}
\newcommand{\CSLRightInline}[1]{\parbox[t]{\linewidth - \csllabelwidth}{#1}\break}
\newcommand{\CSLIndent}[1]{\hspace{\cslhangindent}#1}

\title{Clan, Class or State (Media)?: Evaluating the Roots of Kazakh Bride Abduction}
\author{Marika Olijar\textsuperscript{1}}
\date{}


\shorttitle{Clan, Class or State (Media)?}

\authornote{

PhD Student at University of Wisconsin-Madison in Comparative Politics

Correspondence concerning this article should be addressed to Marika Olijar. E-mail: \href{mailto:olijar@wisc.edu}{\nolinkurl{olijar@wisc.edu}}

}

\affiliation{\vspace{0.5cm}\textsuperscript{1} University of Wisconsin-Madison}

\abstract{
What is the state media's role in reinforcing traditionalism? What implications does traditionalization have for society and particularly violence against women? In the case of Kazakhstan, Kazakhstan's state media plays a role in reinforcing societal traditionalism. These policies manifest in the form of violence against women: non-consensual bride abductions have been steadily increasing in prevalence since the Soviet collapse. I argue that Kazakhstan's re-traditionalization policies could have reinforced the salience of clan survival rather than class mobility. I build on both Kathleen Collin's work on Central Asian clan politics and Cynthia Werner's exploration of economic incentives for men to kidnap wives (Werner (2009)). Collins identifies how clans cut through class cleavages (Collins (2007), 27) and provide a means for social mobility via marriage (Collins (2007), 26). As the the first iteration of this project, I will generate descriptive statistics from an analysis of Central Asian Barometer data as well as International Crime Victims Survey (ICVS) data for Kazakhstan. This discriptive analysis will better prepare me to conduct a content analysis (searching for words associated with clan or class) of state-media and independent media representation of bride kidnapping in Kazakhstan. If clan salience is a better explanation than economic motives, then mitigating the practice of non-consensual bride abduction would not be a matter of solely targeting economic inequality. To what extent is women's political participation or political behavior affected by one's position in society and one's trauma? Since bride kidnapping is a patterned behavior that renders the abducted woman's choice to marry obsolete, woman's choices in other spheres of society may also be impacted.
}



\begin{document}
\maketitle

\hypertarget{introduction}{%
\subsection{Introduction}\label{introduction}}

Since the Soviet Collapse, Kazakh bride kidnapping has been increasing in prevalance Werner (2009). Scholars have argued that economic motives may be at the heart of these kidnappings, with men kidnapping wives to secure social mobility (Werner (2009)). However, it is possible that clans cut through class cleavages (Collins (2007), 27) and provide a means for social mobility via marriage (Collins (2007), 26).
As the the first iteration of this project, I will generate descriptive statistics and an analysis from Central Asian Barometer data as well as International Crime Victims Survey (ICVS) data for Kazakhstan in 2018 to problematize the idea that economic motives are at the heart of bride abduction. I will, first, analyze how pervasive the practice is in Kazakhstan using the ICVS data. As Werner (2009) looks at young unmarried men in particular, my first steps will evaluate the economic grievances among this group, compared to women, and then the whole population from the Central Asia Barometer. Following that analysis, I will determine which media agencies I should look at in further versions of this study.
This descriptive analysis will better prepare me to conduct a content analysis (searching for words associated with clan or class) of state-media and independent media representation of bride kidnapping in Kazakhstan. If clan salience is a better explanation than economic motives, then mitigating the practice of non-consensual bride abduction would not be a matter of solely targeting economic inequality. More broadly, my findings will also speak to literature on state measures in relation to informal institutions.

\hypertarget{summary-statistics-for-the-icvs-data}{%
\subsection{Summary Statistics for the ICVS Data}\label{summary-statistics-for-the-icvs-data}}

\includegraphics{/Users/marikaolijar/Desktop/sample_folder/Final Project/bk_n_2.png}
Legend:
A7: Worry level for bride kidnapping
D0: If married, have you been subject to bride kidnapping?
F1: Male or Female
F2\_1: Age Range
F2: Age
D1: Did you agree to kidnapping in advance? (If D0, yes)
D2: Did you report to the police?
F11: Regions of Kazakhstan

\hypertarget{concern-for-victims-of-kidnapping}{%
\subsection{Concern for Victims of Kidnapping}\label{concern-for-victims-of-kidnapping}}

\includegraphics{/Users/marikaolijar/Desktop/sample_folder/Final Project/A7worry.png}
Ultimately, those surveyed mostly did not worry about relatives being subject to bride kidnapping. However, this is not disaggregated by gender.

\includegraphics{/Users/marikaolijar/Desktop/sample_folder/Final Project/A7genderplot.png}
When we disaggregate the data by gender, we see that women are ``very worried'' more often than men. Of course, the majority of those interviewed are still not that worried about the practice. One thing to note is that the traditional practice has two forms: one with consent and one without consent. It is possible that the people surveyed are not envisioning the non-consensual practice. It is also possible that both types of the practice are being considered by respondents, which could also lead to underreporting of worrying. Next, I look at region, to see if perhaps the prevalence of bride kidnapping is more severe in certain regions of Kazakhstan.

\hypertarget{kidnapping-risk}{%
\section{Kidnapping Risk}\label{kidnapping-risk}}

\includegraphics{/Users/marikaolijar/Desktop/sample_folder/Final Project/bkregion.png}
In Astana (now Nur-Sultan) and Almaty cities, more married women have been subject to bride kidnapping. Almaty oblast and South-Kazakhstan oblast have the next highest incidences of the practice.

\hypertarget{summary-statistics-for-the-central-asia-barometer-data}{%
\subsection{Summary Statistics for the Central Asia Barometer Data}\label{summary-statistics-for-the-central-asia-barometer-data}}

\includegraphics{/Users/marikaolijar/Desktop/sample_folder/Final Project/cab_1.png}
\#\# What do Respondents Report About Their Material Well-Being Compared to Women?
\includegraphics{/Users/marikaolijar/Desktop/sample_folder/Final Project/access_cab.png}
This is an aggregate pool of responses to questions pertaining to material and resource security. The highest ammount of responses is ``somewhat concerned'' to ``very concerned'' about access to food. Food security can be an indicator of economic security. Most respondents have adequate access to electricity, food, and water. Access versus worries over access are viewed differently in the analysis (i.e.~The store may be accross the street, but can you afford the products inside?). Next, I will disagreggate by gender, and pool the above responses as a ``resource marker,'' keeping in mind that access to food is on its own scale for this analysis.

\hypertarget{our-resource-marker}{%
\subsection{Our ``Resource Marker''}\label{our-resource-marker}}

\includegraphics{/Users/marikaolijar/Desktop/sample_folder/Final Project/gender_cab_1.png}
Ultimately, women feel less materially secure than men when it comes to access to electricity, food, and water. Anxiety over food access is also felt significantly more by women. Our ``resource marker'' shows that women in Kazakhstan feel less economically secure than men.

\hypertarget{what-does-media-consumption-look-like-in-kazakhstan}{%
\subsection{What does media consumption look like in Kazakhstan?}\label{what-does-media-consumption-look-like-in-kazakhstan}}

\includegraphics{/Users/marikaolijar/Desktop/sample_folder/Final Project/media.png}
The main news consumed comes in Kazakhstan from the internet and Kazakhstan's national television stations. It is also the most trusted content. kazakhs also watch Russian television stations. Newspapers are mostly to somewhat trusted. This media consumption is consistent with my claim that the state media may be a good vessel for Kazakh re-traditionalization mechanisms. Since some news comes from the internet, it may mean that more independent media may be available, and it would be worth it to deduce whether variation exists between state-run and independent media in terms of representation of clan and class in media articles. It is also important to note that the Kazakh government \href{https://www.opendemocracy.net/en/odr/internet-censorship-in-kazakhstan/}{has been increasing internet restrictions since 2017, even initiating internet shutdowns} {``Internet Censorship in {Kazakhstan}''} (n.d.).

\hypertarget{methods}{%
\section{Methods}\label{methods}}

I report how I determined my sample size, all data exclusions (if any), all manipulations, and all measures in the study in the summary statistics.

I used ICVS data from a 2018 survey conducted in Kazakhstan by Dijk, Van Kesteren, Trochev, and Slade (2018). The Central Asia Barometer Data comes from 2019.

\hypertarget{procedure}{%
\subsection{Procedure}\label{procedure}}

I generated summary statistics to analyze both the Central Asian Barometer data as well as International Crime Victims Survey (ICVS) data for Kazakhstan. I created various data visualizations of descriptive statistics to probe the existing theory about economic motives and bride abduction. I analyzed young men's motives and bride kidnapping's prevalence. I also determined what kind of media agencies I should look at, so that I may scrape words associated with class and clan in later iterations of this study.

\hypertarget{data-analysis}{%
\subsection{Data analysis}\label{data-analysis}}

I used R (Version 4.1.2; R Core Team, 2021) and the R-packages \emph{dplyr} (Version 1.0.7; Wickham, François, Henry, \& Müller, 2021), \emph{forcats} (Version 0.5.1; Wickham, 2021a), \emph{foreign} (Version 0.8.81; R Core Team, 2020), \emph{ggplot2} (Version 3.3.5; Wickham, 2016), \emph{gtsummary} (Version 1.5.0; Sjoberg, Whiting, Curry, Lavery, \& Larmarange, 2021), \emph{papaja} (Version 0.1.0.9997; Aust \& Barth, 2020), \emph{purrr} (Version 0.3.4; Henry \& Wickham, 2020), \emph{readr} (Version 2.0.2; Wickham \& Hester, 2021), \emph{stringr} (Version 1.4.0; Wickham, 2019), \emph{tibble} (Version 3.1.5; Müller \& Wickham, 2021), \emph{tidyr} (Version 1.1.4; Wickham, 2021b), \emph{tidyverse} (Version 1.3.1; Wickham et al., 2019), and \emph{tinylabels} (Version 0.2.1; Barth, 2021) for all my analyses.

\hypertarget{results}{%
\section{Results}\label{results}}

I did not find that respondents worried much about bride kidnapping, unless they were women. Certain regions have higher incidences of bride kidnapping. Access to resources like electricity, water, and food is adequate, but worries and anxiety over food access is present. Again, the worrying effect grows once the resource data is pooled then disaggregated by gender. Next, media consumption in Kazakhstan aligns well with my proposed course of action for media analysis and web-scraping.

\hypertarget{discussion}{%
\section{Discussion}\label{discussion}}

The descriptive statistics do not show correlations between men and economic concerns. Perhaps class is not a sure and solid variable through which to mediate the bride kidnapping practice. Further work should be done to conduct the content analysis suggested in the abstract and eventually infer causality.
\newpage

\hypertarget{references}{%
\section{References}\label{references}}

\begingroup
\setlength{\parindent}{-0.5in}
\setlength{\leftskip}{0.5in}

\hypertarget{refs}{}
\begin{CSLReferences}{1}{0}
\leavevmode\hypertarget{ref-R-papaja}{}%
Aust, F., \& Barth, M. (2020). \emph{{papaja}: {Prepare} reproducible {APA} journal articles with {R Markdown}}. Retrieved from \url{https://github.com/crsh/papaja}

\leavevmode\hypertarget{ref-R-tinylabels}{}%
Barth, M. (2021). \emph{{tinylabels}: Lightweight variable labels}. Retrieved from \url{https://github.com/mariusbarth/tinylabels}

\leavevmode\hypertarget{ref-collins2007}{}%
Collins, K. (2007). \emph{Clan {Politics} and {Regime} {Transition} in {Central} {Asia}}. Cambridge: Cambridge University Press. \url{https://doi.org/10.1017/CBO9780511510014}

\leavevmode\hypertarget{ref-ICVS}{}%
Dijk, J., Van Kesteren, J., Trochev, A., \& Slade, G. (2018). \emph{Criminal victimization in kazakhstan}.

\leavevmode\hypertarget{ref-R-purrr}{}%
Henry, L., \& Wickham, H. (2020). \emph{Purrr: Functional programming tools}. Retrieved from \url{https://CRAN.R-project.org/package=purrr}

\leavevmode\hypertarget{ref-zotero-120}{}%
Internet censorship in {Kazakhstan}: More pervasive than you may think. (n.d.). Retrieved from \url{https://www.opendemocracy.net/en/odr/internet-censorship-in-kazakhstan/}

\leavevmode\hypertarget{ref-R-tibble}{}%
Müller, K., \& Wickham, H. (2021). \emph{Tibble: Simple data frames}. Retrieved from \url{https://CRAN.R-project.org/package=tibble}

\leavevmode\hypertarget{ref-R-foreign}{}%
R Core Team. (2020). \emph{Foreign: Read data stored by 'minitab', 's', 'SAS', 'SPSS', 'stata', 'systat', 'weka', 'dBase', ...} Retrieved from \url{https://CRAN.R-project.org/package=foreign}

\leavevmode\hypertarget{ref-R-base}{}%
R Core Team. (2021). \emph{R: A language and environment for statistical computing}. Vienna, Austria: R Foundation for Statistical Computing. Retrieved from \url{https://www.R-project.org/}

\leavevmode\hypertarget{ref-R-gtsummary}{}%
Sjoberg, D. D., Whiting, K., Curry, M., Lavery, J. A., \& Larmarange, J. (2021). Reproducible summary tables with the gtsummary package. \emph{{The R Journal}}, \emph{13}, 570--580. \url{https://doi.org/10.32614/RJ-2021-053}

\leavevmode\hypertarget{ref-werner2009}{}%
Werner, C. (2009). Bride {Abduction} in {Post}-{Soviet} {Central} {Asia}: {Marking} a {Shift} {Towards} {Patriarchy} through {Local} {Discourses} of {Shame} and {Tradition}. \emph{The Journal of the Royal Anthropological Institute}, \emph{15}(2), 314--331. Retrieved from \url{http://www.jstor.org/stable/20527710}

\leavevmode\hypertarget{ref-R-ggplot2}{}%
Wickham, H. (2016). \emph{ggplot2: Elegant graphics for data analysis}. Springer-Verlag New York. Retrieved from \url{https://ggplot2.tidyverse.org}

\leavevmode\hypertarget{ref-R-stringr}{}%
Wickham, H. (2019). \emph{Stringr: Simple, consistent wrappers for common string operations}. Retrieved from \url{https://CRAN.R-project.org/package=stringr}

\leavevmode\hypertarget{ref-R-forcats}{}%
Wickham, H. (2021a). \emph{Forcats: Tools for working with categorical variables (factors)}. Retrieved from \url{https://CRAN.R-project.org/package=forcats}

\leavevmode\hypertarget{ref-R-tidyr}{}%
Wickham, H. (2021b). \emph{Tidyr: Tidy messy data}. Retrieved from \url{https://CRAN.R-project.org/package=tidyr}

\leavevmode\hypertarget{ref-R-tidyverse}{}%
Wickham, H., Averick, M., Bryan, J., Chang, W., McGowan, L. D., François, R., \ldots{} Yutani, H. (2019). Welcome to the {tidyverse}. \emph{Journal of Open Source Software}, \emph{4}(43), 1686. \url{https://doi.org/10.21105/joss.01686}

\leavevmode\hypertarget{ref-R-dplyr}{}%
Wickham, H., François, R., Henry, L., \& Müller, K. (2021). \emph{Dplyr: A grammar of data manipulation}. Retrieved from \url{https://CRAN.R-project.org/package=dplyr}

\leavevmode\hypertarget{ref-R-readr}{}%
Wickham, H., \& Hester, J. (2021). \emph{Readr: Read rectangular text data}. Retrieved from \url{https://CRAN.R-project.org/package=readr}

\end{CSLReferences}

\endgroup


\end{document}
